\documentclass[a4paper]{book}
\usepackage{fullpage}

\usepackage[utf8]{inputenc}
\usepackage[T1]{fontenc}
\usepackage[francais]{babel}

\usepackage{latexsym}
\usepackage{fancyhdr}
\usepackage{makeidx}
\usepackage{graphics}
\usepackage{graphicx}
\usepackage{longtable}
\usepackage{moreverb}
\usepackage{listings}

\newcommand{\altarica}{{\sc AltaRica}}

\begin{document}

\title{Master 1, Conceptions Formelles\\
Projet du module \altarica\\
Synthèse (assistée) d'un contrôleur du niveau d'une cuve}

\date{}

\author{Nom1 \and Nom2 \and Nom3}

\maketitle

\chapter{Le sujet}
\input{tank}

\chapter{Le rapport}
\section{Rôle de la constante {\tt nbFailures} (2 points)}
La constante nbFailures permet d'éviter que le système ne génère plus d'erreurs que de valves présentes.

\section{Résultats avec le contrôleur initial {\tt Ctrl}}

\subsection{Calcul d'un contrôleur}

\subsubsection{Avec 0 défaillance (1 point)}
\lstinputlisting{Res/System0FCtrl.res}
\lstinputlisting{Res/System0FCtrl0F1I.res}
\lstinputlisting{Res/System0FCtrl0F2I.res}
\lstinputlisting{Res/System0FCtrl0F3I.res}
\lstinputlisting{Res/System0FCtrl0F4I.res}
\paragraph{Interprétation des résultats}
A partir de la seconde itération sur le contrôleur, celui-ci ne contient plus de situations redoutées ni de blocages, ce contrôleur est donc suffisamment performant dans le cas ou il n'y a aucune défaillance.

\subsubsection{Avec 1 défaillance (1 point)}
\lstinputlisting{Res/System1FCtrl.res}
\lstinputlisting{Res/System1FCtrl1F1I.res}
\lstinputlisting{Res/System1FCtrl1F2I.res}
\lstinputlisting{Res/System1FCtrl1F3I.res}
\lstinputlisting{Res/System1FCtrl1F4I.res}
\paragraph{Interprétation des résultats}
A partir de la seconde itération, les résultats se stabilisent mais nous ne sommes plus dans le cas de figure ou il n'y a pas de situations redoutées ou de blocages, le contrôleur proposé n'est donc pas suffisamment performant dans le cas ou il y a une défaillance.

\subsubsection{Avec 2 défaillances (1 point)}
\lstinputlisting{Res/System2FCtrl.res}
\lstinputlisting{Res/System2FCtrl2F1I.res}
\lstinputlisting{Res/System2FCtrl2F2I.res}
\lstinputlisting{Res/System2FCtrl2F3I.res}
\lstinputlisting{Res/System2FCtrl2F4I.res}
\paragraph{Interprétation des résultats}
A partir de la quatrième itération, les résultats se stabilisent mais nous ne sommes plus dans le cas de figure ou il n'y a pas de situations redoutées ou de blocages, le contrôleur proposé n'est donc pas suffisamment performant dans le cas ou il y a deux défaillances.

\subsubsection{Avec 3 défaillances (1 point)}
\lstinputlisting{Res/System3FCtrl.res}
\lstinputlisting{Res/System3FCtrl3F1I.res}
\lstinputlisting{Res/System3FCtrl3F2I.res}
\lstinputlisting{Res/System3FCtrl3F3I.res}
\lstinputlisting{Res/System3FCtrl3F4I.res}
\paragraph{Interprétation des résultats}
A partir de la troisième itération, les résultats se stabilisent mais nous ne sommes plus dans le cas de figure ou il n'y a pas de situations redoutées ou de blocages, le contrôleur proposé n'est donc pas suffisamment performant dans le cas ou il y a trois défaillances.

\subsection{Calcul des contrôleurs optimisés (2 points)}
Reste à faire cette partie

\section{Rôle des composants {\tt ValveVirtual} et {\tt CtrlVV} (4 points)}
Ces composants permettent une gestion plus fine des flots de valves, cela permet de savoir quelle vanne est tombée panne.

\section{Résultats avec le contrôleur initial {\tt CtrlVV}}

\subsection{Calcul d'un contrôleur}

\subsubsection{Avec 0 défaillance (1 point)}
\lstinputlisting{Res/System0FCtrlVV.res}
\lstinputlisting{Res/System0FCtrlVV0F1I.res}
\lstinputlisting{Res/System0FCtrlVV0F2I.res}
\lstinputlisting{Res/System0FCtrlVV0F3I.res}
\lstinputlisting{Res/System0FCtrlVV0F4I.res}
\paragraph{Interprétation des résultats}
A partir de la seconde itération sur le contrôleur, celui-ci ne contient plus de situations redoutées ni de blocages, ce contrôleur est donc suffisamment performant dans le cas ou il n'y a aucune défaillance.

\subsubsection{Avec 1 défaillance (1 point)}
\lstinputlisting{Res/System1FCtrlVV.res}
\lstinputlisting{Res/System1FCtrlVV1F1I.res}
\lstinputlisting{Res/System1FCtrlVV1F2I.res}
\lstinputlisting{Res/System1FCtrlVV1F3I.res}
\lstinputlisting{Res/System1FCtrlVV1F4I.res}
\paragraph{Interprétation des résultats}
A partir de la quatrième itération sur le contrôleur, celui-ci ne contient plus de situations redoutées ni de blocages, ce contrôleur est donc suffisamment performant dans le cas ou il y a une défaillance.

\subsubsection{Avec 2 défaillances (1 point)}
\lstinputlisting{Res/System2FCtrlVV.res}
\lstinputlisting{Res/System2FCtrlVV2F1I.res}
\lstinputlisting{Res/System2FCtrlVV2F2I.res}
\lstinputlisting{Res/System2FCtrlVV2F3I.res}
\lstinputlisting{Res/System2FCtrlVV2F4I.res}
\paragraph{Interprétation des résultats}
A partir de la troisième itération sur le contrôleur, celui-ci ne contient plus de situations redoutées ni de blocages, ce contrôleur est donc suffisamment performant dans le cas ou il y a deux défaillances.

\subsubsection{Avec 3 défaillances (1 point)}
\lstinputlisting{Res/System3FCtrlVV.res}
\lstinputlisting{Res/System3FCtrlVV3F1I.res}
\lstinputlisting{Res/System3FCtrlVV3F2I.res}
\lstinputlisting{Res/System3FCtrlVV3F3I.res}
\lstinputlisting{Res/System3FCtrlVV3F4I.res}
\paragraph{Interprétation des résultats}
A partir de la seconde itération sur le contrôleur, celui-ci ne contient plus de situations redoutées ni de blocages, ce contrôleur est donc suffisamment performant dans le cas ou il y a trois défaillances.

\subsection{Calcul des contrôleurs optimisés (2 points)}
Dans chaque cas de figure, nous arrivons à obtenir des contrôleurs optimisés.

\section{Conclusion (2 points)}

\end{document}
